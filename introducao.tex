\chapter{Introdução} \label{introducao}
\section{Contextualização}
Satélite é um termo utilizado para qualquer objeto que gira em torno de um corpo celeste em função da ação da força gravitacional. Ao longo dos anos, com o avanço da tecnologia, a humanidade desenvolveu e construiu centenas de milhares de satélites artificiais para cumprir diversos tipos de missões, por exemplo, aquisição de dados, monitoramento, posicionamento, telecomunicações, demonstrações, estudos meteorológicos, entre outros. Diariamente, milhões de pessoas consomem serviços que estão, direta ou indiretamente, relacionados com o funcionamento de algum satélite, não há dúvidas que os satélites artificiais permitiram que o mundo se tornasse uma grande aldeia.

Satélite artificial é qualquer dispositivo projetado para funcionar no espaço orbital da Terra, sendo que a principal variável a ser definida no desenvolvimento de um satélite é a sua missão, pois ela que guia boa parte das escolhas do projeto - órbita utilizada, formato, funcionalidades, tempo de vida, entre outras. Porém, para que esses dispositivos consigam prover alguma utilidade, todos têm uma infraestrutura em comum, o que inclui comunicação de dados, suprimento de energia, cálculo de posição, isso além das suas funções customizadas.\cite{nasa_comms_article} 

Este trabalho de conclusão de curso é um memorial de projeto de um EPS para o nanossatélite do projeto Alfacrux. O projeto é uma iniciativa de pesquisadores e alunos da Universidade de Brasília (UnB) que irá colocar em órbita baixa (aproximadamente 500 km de altitude) um nanossatélite até o final de 20XX.

O projeto Alfacrux tem como um de seus objetivos desenvolver um nanossatélite de comunicação em ondas curtas, ou seja, será possível trafegar sinais de voz e dados, porém em baixo volume, uma vez que o foco será prover comunicação em longo alcance para atingir áreas remotas. O nanossatélite será operado pela UnB atráves de bases em solo durante sua vida útil estimada de aproximadamente 2 anos, esse projeto é uma plataforma de interesse estratégico de parceiros como o Exercíto Brasileiro, uma vez que ele poderá ser utilizado, por exemplo, para fornecer comunicações em regiões remotas, a exemplo da Amazônia, onde não se tem infraestrutura ou interesse econômico em ofertar o serviço. 

Juntamente com o desenvolvimento do projeto, nos objetivos acadêmico-científicos, deseja-se criar uma base sólida de documentação, do qual esse trabalho faz parte, para consolidar, registrar e propagar a enorme carga de conhecimento gerada pelos participantes do projeto.

\section{Objetivos}
\subsection{Gerais}\label{gerais}
% Descrever a necessidade do projeto da placa, seguindo o formato CubeSat, etc...

O Sistema Elétrico de Potência (EPS) é um subsistema vital para o funcionamento do \textit{CubeSat}, uma vez que ele é responsável por toda a gestão da energia do satélite. Normalmente, ele é dividido em duas partes principais, sendo uma responsável pelo carregamento da bateria atráves de panéis solares e uma segunda parte de que faz a regulação e distribuição das linhas de tensão necessárias para os demais módulos. 

O objetivo geral desse trabalho é desenvolver uma placa eletrônica que atue como EPS do nanossatélite Alfacrux, respeitando as demandas enumeradas abaixo:

\begin{enumerate}
    \item Possuir proteção contra \textit{undervoltage} e \textit{overvoltage} da bateria
    \item Possuir proteção contra \textit{overcurrent} na linhas reguladas
    \item Possuir sistema que maximiza a potência da placa solar, mais detalhes na seção \ref{mppt_revision}
    \item Possuir baterias para manter o satélite alimentado durante os períodos de eclipse
    \item Possuir controle de quais linhas estão ligadas
    \item Possuir interface de comunicação com OBC - Computador de bordo
    \item Permitir desenvolvimentos futuros
    \item Estar em concordância com o padrão \textit{CubeSat}, mais detalhes na seção \ref{cubesat_revision}
\end{enumerate}{}


\subsection{Específicos}\label{especificos}
% Reunião prof. Café: 
% Colocar com deliverables os levantamentos (consumo, processamento, interfaces)
% Entregavel: lista de especificações de projeto
Objetivos específicos (tarefas/deliverables para o TCC 2)
\begin{enumerate}
    \item Levantamento dos requisitos de consumo
    \item Levantamento das tarefas do nanossatélite
    \item Dimensionar a demanda de energia conhecendo o consumo esperado e as tarefas desempenhadas pelos módulos
    \item Levantamento das interfaces necessárias (entre a própria placa e também com outros sistemas do nanossatélite)
    \item Levantamento das condições de operação
    \item Lista de especificações e componentes escolhidos para o projeto
    \item Simulações dos circuitos do EPS
    \item Projeto da PCB
\end{enumerate}
