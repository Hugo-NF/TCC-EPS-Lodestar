%Usar \chapter*{Título do Capítulo} faz com que o capítulo não seja numerado
\chapter{Conclusão} \label{conclusao}

Este trabalho apresentou um estudo de viabilidade e proposta de topologia para o futuro módulo EPS - \textit{Electrical Power System} para os \textit{cubesats} da constelação Alfacrux da Universidade de Brasília (UnB). 

Inicialmente foi realizada uma conceituação básica do tema, apresentando o projeto Alfacrux em seguida do conceito de \textit{cubesat} e seus módulos, como o módulo EPS se encaixa no \textit{cubesat} e qual papel desempenha. Em seguida, foram apresentados os requisitos propostos para o módulo e quais os objetivos desse trabalho diante desses requisitos.

Na revisão téorica, abordou-se os principais conceitos para o entendimento da topologia proposta, apresentou-se a especificação \textit{cubesat}, o funcionamento básico de um painel solar, os dispositivos de conversão DC-DC, a técnica MPPT - \textit{Maximum Peak Point Tracking} e os dois algoritmos mais utilizados para executá-la.

O desenvolvimento inicia apresentando um \textit{background} de missões anteriores de quatro nanossatélites distintos e quais soluções eram utilizadas para os módulos EPS. Após isso, é mostrado o levantamento que foi realizado em termos dos \textit{payloads}, os seus consumos e tarefas desempenhadas por eles e como que esse consumo energético está ligado ao dimensionamento do EPS.

A proposta foi dividida em duas topologias, uma com os componentes discretos, a qual foi simulada e construída, de acordo com os modelos de referência apresentados para placa solar, baterias e EPS, com o conversor DC-DC LM2735Y da Texas Instruments, que tem as vantagens de ser um conversor mais simples, com menos configurações, ocupando pouco espaço de PCB, contudo atingindo uma alta eficiência. Para a segunda topologia com o controlador MPPT, foi apresentado o SPV1040, o mesmo utilizado na missão ESTCube-1, as suas especificações, modos de operação e um circuito projetado para o cenário proposto conforme explicado nos modelos de referência, porém que não foi possível executar simulações pela ausência de modelos SPICE para o componente.

\begin{table}[H]
\centering
\caption{Bill of Materials - LM2735}
\label{lm2735_bom_table}
\begin{tabular}{|l|l|l|c|c|c|} 
\hline
Component            & Manufacturer         & Part Number                               & \multicolumn{1}{l|}{Qtd}    & \multicolumn{1}{l|}{Price (\$)} & \multicolumn{1}{l|}{Footprint ($mm^2$)}  \\ 
\hline
Rfbb                 & Yageo                & RC0201FR-0710KL                           & 1                           & 0.01                            & 2.08                                     \\ 
\hline
Cf                   & MuRata               & GRM1885C1H182JA01J                        & 1                           & 0.02                            & 4.68                                     \\ 
\hline
Rfbt                 & Yageo                & RC0603FR-0730KL                           & 1                           & 0.01                            & 4.68                                     \\ 
\hline
D1                   & Fairchild~           & SS14FL                                    & 1                           & 0.04                            & 11.7                                     \\ 
\hline
Renable              & Yageo                & RC0201FR-0710KL                           & 1                           & 0.01                            & 2.08                                     \\ 
\hline
Cout                 & Kemet                & C0805C106K8PACTU                          & 2                           & 0.06                            & 13.5                                     \\ 
\hline
Cin                  & Kemet                & C0805C106K8PACTU                          & 1                           & 0.03                            & 6.75                                     \\ 
\hline
Converter            & Texas Instruments    & LM2735YMF/NOPB                            & 1                           & 0.73                            & 15.05                                    \\ 
\hline
L1                   & NIC~                 & NPI43C120MTRF                             & 1                           & 0.09                            & 30.74                                    \\ 
\hline
Current Sensor       & Texas Instruments    & INAx180                                   & 1                           & 0.258                           & 4.64                                     \\ 
\hline
uC                   & Texas Instruments    & \textcolor[rgb]{0.2,0.2,0.2}{MSP430F6769} & 1                           & 6.27                            & 280                                      \\ 
\hline
\multicolumn{1}{l}{} & \multicolumn{1}{l}{} &                                           & \multicolumn{1}{l|}{Total:} & 7.528                           & 375.9                                    \\
\cline{4-6}
\end{tabular}
\end{table}

\begin{table}
\centering
\caption{Bill of Materials - SPV1040}
\label{spv1040_specs_table}
\begin{tabular}{|l|l|l|c|c|c|} 
\hline
Component            & Manufacturer         & Part Number        & \multicolumn{1}{l|}{Qtd}    & \multicolumn{1}{l|}{Price (\$)} & \multicolumn{1}{l|}{Footprint ($mm^2$)}  \\ 
\hline
iC                   & STMicroelectronics   & SPV1040T           & 1                           & 1.79                            & 13.2                                     \\ 
\hline
Cin                  & MuRata               & GRM155R60J475ME87  & 6                           & 0.12                            & 28.08                                    \\ 
\hline
Cout                 & MuRata               & GRM32ER71A226KE20L & 2                           & 0.06                            & 9.36                                     \\ 
\hline
L                    & Panasonic~           & ELLATV100M         & 1                           & 0.12                            & 100                                      \\ 
\hline
Rh                   &                      &                    & 1                           & 0.01                            & 2.08                                     \\ 
\hline
RL                   &                      &                    & 1                           & 0.01                            & 2.08                                     \\ 
\hline
CinSNS               &                      &                    & 1                           & 0.03                            & 4.68                                     \\ 
\hline
CoutSNS              &                      &                    & 1                           & 0.03                            & 4.68                                     \\ 
\hline
Dout                 & STMicroelectronics   & SMM4F5.0           & 1                           & 0.03                            & 11.6                                     \\ 
\hline
Rs                   &                      &                    & 1                           & 0.01                            & 2.08                                     \\ 
\hline
Rf1                  &                      &                    & 1                           & 0.01                            & 2.08                                     \\ 
\hline
Rf2                  &                      &                    & 1                           & 0.01                            & 2.08                                     \\ 
\hline
CF                   &                      &                    & 1                           & 0.03                            & 4.68                                     \\ 
\hline
R3                   &                      &                    & 1                           & 0.01                            & 2.08                                     \\ 
\hline
\multicolumn{1}{l}{} & \multicolumn{1}{l}{} &                    & \multicolumn{1}{l|}{Total:} & 2.27                            & 188.76                                   \\
\cline{4-6}
\end{tabular}
\end{table}

No fim, ambas as topologias são capazes de atender os aspectos técnicos necessários para o EPS, conforme observado na tabela \ref{comparison_specs_table} que coloca lado a lado as especificações de ambos os dispositivos, em seguida, as tabelas \ref{lm2735_bom_table} e \ref{spv1040_specs_table} mostram uma possível lista de materiais requeridos por cada uma das implementações.

\begin{table}
\centering
\caption{Especificações dos dispositivos}
\label{comparison_specs_table}
\begin{tabular}{|l|ccc|ccc|c|} 
\cline{2-8}
\multicolumn{1}{c|}{}               & \multicolumn{3}{c|}{LM2735Y}                              & \multicolumn{4}{c|}{SPV1040}                                           \\ 
\hline
\multicolumn{1}{|c|}{Specification} & \multicolumn{1}{c|}{Min} & \multicolumn{1}{c|}{Typ} & Max & \multicolumn{1}{c|}{Min} & \multicolumn{1}{c|}{Typ} & Max & Unit       \\ 
\hline
Input                               & 2.7                      & -                        & 5.5 & 0.3                      & -                        & 5.5 & V          \\ 
\hline
Output                              & 3                        & -                        & 24  & 2                        & -                        & 5.2 & V          \\ 
\hline
Switch current                      & -                        & -                        & 2.1 & -                        & -                        & 1.8 & A          \\ 
\hline
Switching Frequency                 & 360                      & 520                      & 580 & 70                       & 100                      & 130 & kHz        \\ 
\hline
Quiescent Current                   & -                        & 80                       & -   & -                        & 60                       & 80  & nA         \\ 
\hline
Power ON Resistance                 & -                        & 170                      & 330 & -                        & -                        & 120 & m$\Omega$  \\ 
\hline
Efficiency                          & -                        & -                        & 90  & -                        & -                        & 95  & \%         \\ 
\hline
Operating Temperature               & -40                      & -                        & 125 & -40                      & -                        & 125 & ºC         \\
\hline
\end{tabular}
\end{table}

Portanto, visando os critérios de um menor número de componentes, o que, por sua vez, diminui o esforço para integração e de programação de microcontroladores, um espaço menor de espaço de PCB e um menor custo de implementação, a topologia com o circuito integrado de controlador MPPT é a mais recomendada.

\subsection*{Recomendações e Próximos Passos}

Os trabalhos futuros são de extrema importância para a continuidade do projeto, uma continuação lógica dá-se na aquisição dos componentes aqui apresentados e estudados para a realização de uma prova de conceito e validar o que foi proposto, eliminando todas as restrições impostas pela simulação. Outra tarefa que também deve vir em seguida é a integração desses componentes que compõem o circuito de carga da bateria ao \textit{layout} de PCB para o módulo EPS que já se encontra em fase de projeto e desenvolvimento pelo aluno Luiz Antônio, com a orientação do Professor Daniel Café, trabalho o qual estou acompanhando de perto.

