%Usar \chapter*{Título do Capítulo} faz com que o capítulo não seja numerado
\chapter{Conclusão} \label{conclusao}

Este trabalho apresentou um estudo de viabilidade e proposta de topologia para o futuro módulo EPS - \textit{Electrical Power System} para os \textit{cubesats} da constelação Alfacrux da Universidade de Brasília (UnB). 

Inicialmente foi realizada uma conceituação básica do tema, apresentando o projeto Alfacrux em seguida do conceito de \textit{cubesat} e seus módulos, como o módulo EPS se encaixa no \textit{cubesat} e qual papel desempenha. Em seguida, foram apresentados os requisitos propostos para o módulo e quais os objetivos desse trabalho diante desses requisitos.

Na revisão téorica, abordou-se os principais conceitos para o entendimento da topologia proposta, apresentou-se a especificação \textit{cubesat}, o funcionamento básico de um painel solar, os dispositivos de conversão DC-DC, a técnica MPPT - \textit{Maximum Peak Point Tracking} e os dois algoritmos mais utilizados para executá-la.

O desenvolvimento inicia apresentando um \textit{background} de missões anteriores de quatro nanossatélites distintos e quais soluções eram utilizadas para os módulos EPS. Após isso, é mostrado o levantamento que foi realizado em termos dos \textit{payloads}, os seus consumos e tarefas desempenhadas por eles e como que esse consumo energético está ligado ao dimensionamento do EPS.

A proposta foi dividida em duas topologias, uma com os componentes discretos, na qual foram simulados e comparados vários modelos de conversores DC-DC, seguindo os modelos de referência apresentados para placa solar, baterias e EPS. O primeiro deles o LM2735 com a proposta de ser uma solução mais simples, com menos configurações, menos espaço de PCB, porém com bastante eficiência. O LT8364 como uma solução mais maleável em termos de configurações. Para a segunda topologia com o controlador MPPT, foi apresentado o SPV1040, o mesmo utilizado na missão ESTCube-1, as suas especificações, um circuito projetado para o cenário proposto conforme explicado nos modelos de referência, porém que não foi possível executar simulações pela ausência de modelos SPICE para o componente.

\subsection*{Recomendações e Próximos Passos}

Os trabalhos futuros são de extrema importância para a continuidade do projeto, uma continuação lógica dá-se na aquisição dos componentes aqui apresentados e estudados para a realização de uma prova de conceito e validar o que foi proposto, eliminando todas as restrições impostas pela simulação. Outra tarefa que também deve vir em seguida é a integração desses componentes que compõem o circuito de carga da bateria ao \textit{layout} de PCB para o módulo EPS que já se encontra em fase de projeto e desenvolvimento pelo aluno Luiz Antônio, com a orientação do Professor Daniel Café, trabalho o qual estou acompanhando de perto.

