\chapter*{Resumo}
O presente trabalho apresenta um estudo de viabilidade e ao final propõe uma versão para o desenvolvimento de um dispositivo EPS, sigla para \textit{Electrical Power System}, para um \textit{Cubesat} 1U. O objetivo consiste em propor uma placa eletrônica que realize a aquisição, armazenamento e distribuição de energia, além de atender os demais requisitos da missão em diversas frentes, tais como a padronização, as conexões com os sistemas embarcados necessários, requisitos de operação, consumo, entre outros. Será apresentado em detalhes um \textit{background} com missões anteriores, de outras universidades do mundo inteiro, alguns conceitos teóricos necessários e, por fim, o hardware proposto para o projeto. Esse trabalho está inserido no contexto do projeto Alfacrux, desenvolvido pelo laboratório LODESTAR, Laboratório de Simulação e Controle de Sistemas Aeroespaciais, da Universidade de Brasília (UnB), que tem como um de seus objetivos desenvolver uma constelação de nanossatélites, sendo o primeiro lançamento previsto para até o final de 2023. 

\textbf{\textit{Keywords}: Cubesat, nanossatélite, EPS, MPPT, microcontrolador, conversão, DC-DC}

% PENDING: Revisão do Café